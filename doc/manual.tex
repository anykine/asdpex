\documentclass[10pt,a4paper]{article}
\usepackage[utf8]{inputenc}
\usepackage[english]{babel}
\usepackage{amsmath}
\usepackage{amsfonts}
\usepackage{amssymb}
\usepackage{graphicx}
\usepackage{lmodern}
\usepackage{hyperref}
\usepackage[left=2cm,right=2cm,top=2cm,bottom=2cm]{geometry}
\author{Marten Jäger}
\title{Manual}
\begin{document}
\maketitle

For several regions of the new genome release GRCh38 exists alternative loci.

The reference genome chromosomes with their corresponding accession numbers are
stored in \\

\url{ftp://ftp.ncbi.nlm.nih.gov/genomes/H_sapiens/Assembled_chromosomes/chr_accessions_GRCh38.p2}

e.g. it looks like this:
\begin{verbatim}
#Chromosome	RefSeq Accession.version	RefSeq gi	GenBank Accession.version	GenBank gi
1	NC_000001.11	568815597	CM000663.2	568336023
2	NC_000002.12	568815596	CM000664.2	568336022
...
MT	NC_012920.1	251831106	J01415.2	113200490
\end{verbatim}

An overview of the alternative loci can be found in \\

\url{ftp://ftp.ncbi.nlm.nih.gov/genomes/H_sapiens/Assembled_chromosomes/alts_accessions_GRCh38.p2}

\begin{verbatim}
#Chromosome	RefSeq Accession.version	RefSeq gi	GenBank Accession.version	GenBank gi
1	NW_011332688.1	732663890	KN538361.1	729255073
1	NW_009646195.1	698040070	KN196473.1	693592224
1	NW_009646194.1	698040071	KN196472.1	693592225
1	NW_009646196.1	698040069	KN196474.1	693592223
1	NW_011332687.1	732663891	KN538360.1	729255091
2	NW_011332690.1	732663888	KN538363.1	729255051
...
\end{verbatim}

Where the Chromosome column corresponds to the Chromosome column in the
\textit{chr\_accessions\_GRCh38.p2} file and the \textit{GenBank
Accession.version} column in modofied version to the chromosome ref in the VCF
files.


The regions are defined in a file at \\
\url{ftp://ftp.ncbi.nlm.nih.gov/genomes/H\_sapiens/chr\_context\_for\_alt\_loci/GRCh38.p2/genomic\_regions\_definitions.txt}


\begin{verbatim}
#region_name    chromosome      start   stop
REGION108       NC_000001.11    2448811 2791270
PRAME_REGION_1  NC_000001.11    13075113        13312803
REGION200       NC_000001.11    17157487        17460319
REGION189       NC_000001.11    26512383        26678582
REGION109       NC_000001.11    30352191        30456601
FOXO6   NC_000001.11    41250328        41436604
REGION190       NC_000001.11    112909422       113029606
CEN1    NC_000001.11    122026460       125184587
1Q21    NC_000001.11    144488706       144674781
...
\end{verbatim}


Which alt loci belongs to which region can now be derived from the files in the
subfolder of  \\
\url{ftp://ftp.ncbi.nlm.nih.gov/genomes/H_sapiens/chr_context_for_alt_loci/GRCh38.p2/ALT_REF_LOCI_X/alt_scaffolds/alt_scaffold_placement.txt}

where \textit{X} is in $[1..35]$.

This means we can build up a tree like structure for the alt loci

\begin{verbatim}
genome
|- chr1
|- chr2
:   |-Region2.1
:   |-Region2.2
:   :   |-alt_loci2.2.1
:   :   |-alt_loci2.2.2
:   :   |-alt_loci2.2.3
:   :   :
\end{verbatim}


\subsection*{Regions}

\begin{itemize}
\item 178 regions with alternativ loci
\item $1 - 35$ alt. loci per region
\item cumulative length: 173.055.655bp
\item assuming the genome is of size 3.088.269.832bp (only top chromosomes), the alt loci cover about $5.6\%$ of the whole genome
\end{itemize}

\subsection*{alt. loci}

\begin{itemize}
\item 261 alt. loci
\item there are varying numbers of alt. loci covering a region. A alt. loci does not have to cover the whole region (e.g 1Q21, ADAM5, APOBEC, ... - mostly single alt. loci <-> region relations) but can also cover only a part of the region (e.g. ABR, CYP2D6, ...). Some regions are even not covered completely by alt. loci but are defined by the most 5' and 3' alt loci ends in a specific genomic range (e.g. KRTAP\_REGION\_1, OLFACTORY\_REGION\_1, PRADER\_WILLI, ...)
\end{itemize}

\subsection*{seeds}

There exists alignment information files for each alt. loci on the NCBI FTP-server. These alignments are stored in GTF format (single row) and 

\subsection*{alignment}

\begin{itemize}
\item 
\end{itemize}

\section*{MANUAL}

\subsection*{Usage}

\paragraph*{download}

\begin{verbatim}
java -jar hg38altlociselector-cli-0.0.1-SNAPSHOT.jar download GRCh38
\end{verbatim}

Using the \texttt{download} command, all necessary files are downloaded. This includes the genome reference (from BWA-kit) and the region and alt loci definition files (from NCBI s.o.).

\paragraph*{create-fa}
\begin{verbatim}
java -jar hg38altlociselector.jar create-fa -o data/
\end{verbatim}

Creates fasta files for the regions. The alt loci can be extended and adapted to the strand of the reference. Using the \texttt{-o} flag, the outputfolder can be specified.

\paragraph*{create-seed}
\begin{verbatim}
java -jar hg38altlociselector.jar create-seed -o data/
\end{verbatim}

Creates seed files used by the Seqan-Alignment tool in the specified output folder.

\paragraph*{align}
\begin{verbatim}
java -jar hg38altlociselector.jar align -d data/ -s
\end{verbatim}

Creates the Fasta and seed files and stores them in the \textit{temporary} folder. then calls the Seqan aligner and that creates the VCF files.\\
With the \texttt{-d} flag you can specify the data directory and with \texttt{-s} you will generate single files for each 

\paragraph*{annotate}
\begin{verbatim}
java -jar hg38altlociselector.jar annotate -v file.vcf.gz -a alt_loci.vcf.gz
\end{verbatim}

The \texttt{annotate} command will annotate an existing vcf-file with the help and knowledge about alternative loci. It will take all known alternative loci for a specific region and according to the overlap with SNVs between the reference toplevel allele and the alternative loci alleles, decide which allele is the most probable. Those variants which are FP in the selected allele will be marked in the \textit{FILTER} column.
\begin{verbatim}
##FILTER=<ID=altloci,Description="This is a FP variant according to the most probable alt loci.">
\end{verbatim}


There are several checks to see if the variant distribution indicates a specific common allele.
\begin{enumerate}
\item Imaging, we have two sets of variants $\mathcal{A}$ with the SNVs defining the difference between toplevel chromosome allele and alt loci allele and $\mathcal{B}$ which are those variants found in the specific sample $\mathcal{S}$ for the same chromosomal region $\Re$.\\
The 
\end{enumerate}

\end{document}